
\documentclass{article}
\usepackage[utf8]{inputenc}

\title{Summery on Philosophy of  Artificial Intelligence}

\author{Bhavin Raichura}

\date{July 2021}
\begin{document}

\maketitle

\begin{keyword}
keywords- Intelligence, Ethics, Consciousness, Epistemology, Free will, Thinking, Turing test, Mind, Mental state, Human brain.

\end{keyword}

\section{Introduction}
The Philosophy of AI is a field that explores artificial intelligence and its implications for the understanding of knowledge and intelligence, ethics, consciousness, epistemology and free will, it is use to answer of issues related to whether the things in AI is possible or not. Some of the following are:
Whether or not it is possible to build an intelligent thinking machine like human? 
And if it is possible then the machine have same intelligence like human or not? 
whether or not it is possible to build a machine that can have a mind, mental states, and consciousness in the same sense that a human being can? Can it feel how things are?
From above we can conclude that the key problem is understanding common sense knowledge and abilities. 



\section{Can a machine display general intelligence?}
Many researchers make their own statements of machine intelligence. But arguments suggest that building a working AI system is impossible because there is some practical limit to the capabilities of a computer or that there is some special quality of the human mind that is necessary for intelligent behavior.

Alan Turing proposed that: 

"If a machine behaves as intelligently as a human being, then it is as intelligent as a human being."

He comes up with the Turing test where, in an online chat room, a real person interacts with a computer program. The program passes the test if one cannot tell which of the two participants is human.
But the Turing test is that it only measures the "humanity" of a machine's behavior, rather than the "intelligence" of behavior. Since human behavior and intelligent behavior are not exactly the same thing, the test fails to measure intelligence.

\section{Symbol processing}
Symbol processing is the ability to visualize and define objects by their symbolic characteristic. This claim is very strong because it implies that human thinking is a form of symbol manipulation (since a symbol system is necessary for intelligence) and that machines can be intelligent (since a symbol system is sufficient for intelligence).


Allen Newell and Herbert A. Simon's physical symbol system hypothesis: 

"A physical symbol system has the necessary and sufficient means of general intelligent action."


Those "symbols" that they discussed are higher level - symbols that directly correspond to objects in the world, such as dog and tail.Further arguments suggest that human thinking does not involve (only) high-level symbol manipulation. They do not show that artificial intelligence is impossible, requiring more than just symbol processing. 

\section{Can a machine have a mind, consciousness, and mental states?}

The words mind and consciousness are used by different communities in different ways. For others , the words mind or consciousness are used as a kind of secular synonym for the soul. It's not hard to give a commonsense definition of "consciousness" observes philosopher John Searle. Some of the harshest critics of artificial intelligence agree that the brain is just a machine, and that consciousness and intelligence are the result of physical processes in the brain.

We are in early stage of AI where we use some algorithms of weak Al. The terms "Week AI" and "Strong AI" are defined by Searle. Weak AI is artificial intelligence that implements a limited part of mind, or, as narrow AI, is focused on one narrow task, where Strong AI is an artificial intelligence construct that has mental capabilities and functions that mimic the human brain and also have feeling. It is very difficult for now to develop a machine have a same  mind and consciousness.  A significant amount of research is needed here to understand how the human brain system may have developed consciousness, and if so, how can it be theoretically proven that it is exhibiting such a phase. Scientists have been offered various theories, none of them has yet passed the hypothetical stage and has not provided valuable evidence.



\end{document}

\footnote{intelligence}
\footnote{ethics}
\footnote{consciousness}
\footnote{epistemology}
\footnote{free will}
\footnote{thinking}
\footnote{turing test}
\footnote{mind}
\footnote{mental state}
\footnote{human brain}


