\documentclass[10pt,a4paper,twoside]{article}
\usepackage[dutch]{babel}
\usepackage{amssymb}
\usepackage{amsmath}
\usepackage{float,flafter}	
\usepackage{hyperref}
\usepackage{inputenc}
\setlength\paperwidth{20.999cm}\setlength\paperheight{29.699cm}\setlength\voffset{-1in}\setlength\hoffset{-1in}\setlength\topmargin{1cm}\setlength\headheight{12pt}\setlength\headsep{0cm}\setlength\footskip{1.131cm}\setlength\textheight{25cm}\setlength\oddsidemargin{2.499cm}\setlength\textwidth{15.999cm}

\begin{document}
\begin{center}
\hrule

\vspace{.3cm}
{\bf {\huge Assignment 3}}
\vspace{.3cm}
\end{center}
{\bf Name:}  Bhavin Raichura\\
{\bf Roll no:}  19111019 \\
{\bf Branch: }  Biomedical Engineering \hspace{\fill}  21 July, 2021 \\
\hrule

\vspace{.5cm}
\vspace{.4cm}
{\bf {\Large Moravec's Paradox }}\\
\section{Summery}
Moravec's paradox is an observation by AI and robotics researchers that, contrary to conventional beliefs, logic requires little computation, but sensorimotor skills require enormous computational resources.

\item He wrote that, 
$"$it is comparatively easy to make computers exhibit adult level performance on intelligence tests or playing checkers, and difficult or impossible to give them the skills of a one-year-old when it comes to perception and mobility"

\section{The biological basis of human skills}
One possible explanation of the paradox, offered by Moravec, is based on evolution. All human skills are implemented biologically, using machinery designed by the process of natural selection. In the course of their evolution, natural selection has tended to preserve design improvements and optimizations. The older a skill (like recognizing a face, judging people's motivations, recognizing a voice) is, the more time natural selection has had to improve the design.



\section{Historical influence on AI}
Rodney Brooks explains that, according to early AI research, it is easy for a machine to solve complex problems such as algebra words, but difficult to perform tasks that a young child could easily do.
He decided to build intelligent machines that "had no cognition. This new direction, which he called "Novel AI", is highly influential on robotics research and AI.
\item Some examples of skills that have emerged recently: math, engineering, sports, logic, and scientific reasoning. These are hard on us because they are not what our bodies and minds were primarily developed to do.

\section{Result}
Thirty-five years of AI research shows that difficult problems are easy and easy problems are difficult. For example the mental ability of a four-year-old that we take for granted (like recognizing faces, picking up a pencil) actually solves some of the most difficult engineering problems ever conceived.
\end{document}
