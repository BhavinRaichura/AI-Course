\documentclass[10pt,a4paper,twoside]{article}
\usepackage[dutch]{babel}
\usepackage{amssymb}
\usepackage{amsmath}
\usepackage{float,flafter}	
\usepackage{hyperref}
\usepackage{inputenc}
\setlength\paperwidth{20.999cm}\setlength\paperheight{29.699cm}\setlength\voffset{-1in}\setlength\hoffset{-1in}\setlength\topmargin{1cm}\setlength\headheight{12pt}\setlength\headsep{0cm}\setlength\footskip{1.131cm}\setlength\textheight{25cm}\setlength\oddsidemargin{2.499cm}\setlength\textwidth{15.999cm}

\begin{document}
\begin{center}
\hrule

\vspace{.3cm}
{\bf {\huge Assignment 4}}
\vspace{.3cm}
\end{center}
{\bf Name:}  Bhavin Raichura\\
{\bf Roll no:}  19111019 \\
{\bf Branch: }  Biomedical Engineering \hspace{\fill} 5 August, 2021 \\
\hrule

\vspace{.5cm}
\vspace{.4cm}
{\bf {\Large From pseudoscience to science: pulse diagnosis  }}\\


\section{Introduction}
\item
Pulse diagnosis is a diagnostic technique used in Ayurveda, Traditional Chinese Medicine, Traditional Mongolian Medicine, Siddha Medicine, Traditional Tibetan Medicine, and Unani. Although it once showed many positive results, it no longer has scientific validity, but research continues and is not defined in some derivative texts, and is subjective.
\item
In Ayurveda, advocates claim that by examining the nadi, imbalances in the three doshas (vata, pitta and kapha) can be diagnosed and the balance of prana, tejas and oja can also be determined.
\item
In traditional Chinese medicine, there are several systems (Cun Kou) use composite pulse properties, taking into account changes in the evaluation parameters of the pulse, to obtain one of the traditional 29 pulse types. They are analyzed based on a number of factors including depth, speed, length and fluid level.

\section{Collecting data for diagnosis}
Approaches focus on individual pulse conditions, looking at changes in pulse quality and strength within the condition, with each condition having a relationship to a particular body region. For example, each paired pulse position may represent the upper, middle, and lower cavities of the torso, or be associated individually with specific organs. (For example, the small intestine is said to be reflected in the pulse in the left superficial position and the heart in the deep position.)

\section{How AI can be used}

\begin{itemize}
\item Under the medical world are using Artificial intelligence and machine learning to develop new models. Using pulse diagnosis data we can develop models that can be used to develop new medical diagnostic tools and mobile applications for analysis.
\end{itemize}

\begin{itemize}
\item Artificial intelligence can be used to collect data of patient on how the pulse will react to various medical conditions and than analysis it. After analyzing how accurate the data will be. This analysis data can also be used for diagnosis using telemedicine.
\end{itemize}



\end{document}
