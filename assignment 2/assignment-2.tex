\documentclass{article}
\usepackage[utf8]{inputenc}

\title{Gödel's incompleteness theorems}
\author{Bhavin Raichura, 19111019}
\date{22 July 2021}

\begin {document}

\maketitle


Gödel's incompleteness theorems are two theorems of mathematical logic that deal with the limits of provability in formal axioms. Gödel proved that it is impossible to prove that a formal mathematical system is free from contradiction. So mathematics is unable to give us certainty – it is always possible that one day we will find a contradiction.

\item 
A \textbf{formal system} is a deductive apparatus that consists of a particular set of axioms along with rules of symbolic manipulation that allow for the derivation of new theorems from the axioms.

\item
A formal system is said to be \textbf{effectively axiomatized} if its set of theorems is a recursively enumerable set. A set of \textbf{axioms is complete} if, for any statement in the axioms language, that statement or its negation is provable from the axioms. A set of axioms is \textbf{consistent} if a statement is not such that both the statement and its negation can be proved by the axioms, and are otherwise inconsistent.

\subsection{First incompleteness theorem}
\item \textbf{theorem : }"Any consistent formal system F within which a certain amount of elementary arithmetic can be carried out is incomplete; i.e., there are statements of the language of F which can neither be proved nor disproved in F."
\item He proved that any sufficiently powerful axiomatic system of number theory must be either incomplete (i.e., able to express a true statement that is not a theorem) or inconsistent (i.e., able to express a false statement that is a theorem). 

\subsection{Second incompleteness theorem}
The second incompleteness theorem states that an axiomatic system F cannot “prove its own consistency.”

\item Gödel's second incompleteness theorem is stronger than the first incompleteness theorem because the statement built into the first incompleteness theorem does not directly express the consistency of the system. Also, the proof of the second incompleteness theorem is obtained by formalizing the proof of the first incompleteness theorem within the system.

\end{document}
